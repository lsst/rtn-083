\section{Design Guidance and List of RFL Products \label{RFLproducts}}

\subsection{Key high-level messages \label{khlm}}

Key high-level messages will capture the essence of Rubin Observatory and LSST, and will be
consistent with Rubin's mission statement discussed above. Furthermore, these messages
will adhere to the following principles:
\begin{itemize}
\item Messages should focus on, e.g., why this news should be ``above the NYT fold''
\item There should be no jargon (like ``system'')
\item Messages should make things as understandable as possible for media (sound bites)
\item Include ``familiar equivalences'' that people can understand (e.g., DKIST sees solar structure the size of Texas;
     in Rubin's case, such equivalences and other analogies will be crucial to capture the size and complexity of its dataset) 
\item Use a single name to refer to the observatory, camera and survey. Initial: NSF–DOE Vera C. Rubin Observatory thereafter Rubin. 
\item Refer to other names (LSST Camera, Simonyi Survey Telescope) ONLY when necessary, preferably only in the ``about'' section after the main release).
\end{itemize}

Key high-level messages should be organized in
\begin{enumerate}
\item Primary messages about Rubin First Look
\item Secondary messages about Rubin First Look and more generally about Rubin and LSST
\end{enumerate}

These key high-level messages should be completed (apart from embargoed details about
the actual RFL dataset) well before the RFL event. 
 

\subsection{Images and image-based products}

The most successful past image releases from major telescopes were not only stunning visually, but also told a story scientifically. The team developing this plan and those executing it should remember that it is the scientific potential that captures the media and public attention. The RFL release is not intended to showcase actual publishable science results, but provide the most evocative images possible to illustrate Rubin’s wide-field, high cadence and amazing science potential. Using such proxies is also important as so-called Science press releases will need to be based on peer-reviewed science papers.


There are huge expectations from Rubin stakeholders for this event -- most people are envisioning a splash similar to what JWST and Event Horizon Telescope achieved. Yes, we need to acknowledge that Rubin’s images will not be even remotely as spectacular as those from
JWST (or even from DKIST) because of space-based resolution vs. ground-based resolution difference. Indeed, Rubin Observatory is not designed to produce spectacular images; its purpose is to produce {\bf a very large number of very large 
images} with ground-based (mediocre, by space standards) resolution. In some sense, an ultimate celebratory moment 
will be something like ``LSST’s last photon'' in 2035, rather than first photons and First Look in 2025 because
{\bf it is the final LSST dataset that will be unprecedented and impressive but not Rubin's first light images}. 

Given that fact, the best strategy to attract and excite media rests on two important pillars:
\begin{enumerate} 
\item Do not rely solely on static images but use tools for zoom in/out to convey simultaneously large field of view and many pixels 
\item Showcase time domain aspects and unprecedented Rubin's etendue (about 100 times faster surveying speed than other 8m telescopes) 
\item As an extend goal, introduce Rubin software and AI-powered data interpretation (``astronomers cannot look at so many images, but computers can!'')
\end{enumerate}

The Images Working Group (see Section~\ref{WGs}) is charged with designing and executing an observing
program that will follow this strategic guidance and produce a dataset and data products that will
maximize the success of RFL. 

Note that this discussion intentionally avoids embargoed details (see Section~\ref{embargo}). 


\subsection{Press kit}

In addition to the actual RFL image-based products, an informative, self-contained, detailed and easy to use Press Kit
will be a crucial product to ensure RFL success. Its purpose is to inform the media about Rubin Observatory and provide them with all the material they need to easily write stories and social media posts about Rubin, and ultimately about the RFL images.
In other words, the Press Kit needs to make the media’s job covering Rubin as easy as possible. 
The Press Kit will be created in both English and Spanish.

Many publications no longer have dedicated science writers, so the information in the Rubin Press Kit must be ready
for them to use ``as is'' and should be written for a non-science audience. That means the text should be free of jargon,
and all science concepts must be explained in non-specialist language. In addition, readily understandable comparisons,
analgoies and equivalencies are strongly recommended for the media who will be creating content for general audiences
(e.g., one Rubin image is the size of 45 full moons).

The Press Kit will include: 
\begin{itemize}
\item Press Release text (including embargo information) 
\item Information about and pointers to image-based products and tools, discussed above.
\item Pre-recorded short video interviews with selected Rubin team members and expert scientists (see Section~\ref{interviews}) 
\item Rubin textual background information (see below).
\item Videos about Rubin Observatory and LSST science. 
\item Additional pointers, such as weblinks, to additional more detailed and specialist information (e.g., software pipelines,
           Rubin image gallery, Rubin YouTube channel).
\item Contact list of media-trained cadre of scientists and engineers for interviews on specific topics

\end{itemize}

This Press Kit will be made available for download in a single collated, printable and linked pdf document,
as well as website text that is easy to navigate.


\subsection{Rubin Background Information} 

The Press Kit will also include extensive and appropriate textual background information.

The background information will be organized by general topic, and written in brief sentences that can be used as
sound-bites for in-person interviews. These are similar to key high-level messages discussed above, but are typically
longer and/or more detailed, and can be used as supplementary material to support the key messages.

Topics will include Rubin Observatory, Simonyi Survey Telescope, LSST Camera, Data Facility, LSST Software, LSST survey,
Science, People, Funding, Name, History of Rubin Observatory, Rubin Superlatives. This background information will
be made available in a single collated, printable and linked pdf document, as well as website text that is easy to navigate. 

  