\section{Team Organization, Roles \& Responsibilities, Schedule}

The RFL event is being planned and organized by a team that includes staff members
from many stakeholder institutions with relevant expertise and experience. Early
RFL team organization intially reflected existing organizational Rubin Construction
and Operations staff units, but recently it was reorg-ed to improve coordination
with experts from other organizations, such as AURA, SLAC, NOIRLab's CEE and LSST DA. 


\subsection{Rubin Celebration Organizing Committee}

Rubin Celebration Organizing Committee (RCOC) is
an overarching committee designed to assure input from all key stakeholders, and progress reporting to all stakeholders, in the context of both FL release and two dedication ceremonies (the Simonyi Survey Telescope naming ceremony and the Rubin Observatory dedication ceremony)


\subsection{System First Light Coordination Group}

System First Light Coordination Group (SFLcg) is charged to coordinate the work of three working groups
described below, track their progress, and report it back to the RCOC. For historic reasons, this group's
is named after System First Light rather than Rubin First Look. The SFLcg is chaired by the Construction
Project Director (\v{Z}eljko Ivezi\'{c}). 


\subsection{RFL Working Groups \label{WGs}}

There are three principal working groups charged to design the RFL content, produce it, package it for distribution,
and distribute it to media and public. They include
\begin{itemize}
\item Images WG is charged with planning the observing strategy and processing for the FL images. The IWG is
  chaired by the Rubin Project Scientist (Steve Ritz).
\item EPO WG is charged to produce, or coordinate the production of, all RFL products other than image-based
  products. The EPO WG derives its name from the fact that most of its members are recruited from the homonymous
  unit of the Rubin Operations and it is co-chaired by the EPO Head (Alan Strauss) and the Rubin Science Writer (Kristen Metzger). 
\item Communications WG is charged with further developing and implementing the media engagement strategy
         outlined in this document. It is chaired by the Head of Rubin Communications (Ranpal Gill). 
\end{itemize} 

These working groups include about 30 people who are mostly staff from the Rubin Construction Project and Rubin Operations,
representatives from all stakeholders, and also include experts from AURA HQ and NOIRLab’s CEE with relevant experience and
expertise. The RFL team is a ``badgeless Rubin team'' working together to achieve goals described in this document. 

In order to better illustrate their expected deliverables, here is an incomplete but representative
list for each WG: 

Images WG deliverables:
\begin{itemize}
\item choose the observing strategy and targets  
\item produce processed RFL images and image-based products 
\item add sonification and alt-text (with EPO/Comms teams)  
\item help with narrative for the Press Release 
\item pre-RFL end-to-end tests (e.g., utilizing ComCam as pathfinder) 
\item communicate and coordinate with other relevant teams and committees (in particular,
           with the Rubin Project Science Team and NOIRLab's CEE)
\end{itemize} 


EPO WG deliverables:
\begin{itemize}
\item deliver momentum building “Science Stories” series 
\item produce RFL Press Kit 
\item produce educational video clips about Rubin 
\item deliver sonification and alt-text for RFL images
\end{itemize} 


Communications WG deliverables:
\begin{itemize}
\item distribute momentum building “Science Stories” series
\item produce RFL Press Release
\item distribute RFL Press Kit 
\item develop social network strategy and management
\item lead the media momentum building (including social networks and influencers)
\item lead coordination with ``RFL hubs''  
\item RFL success tracking 
\end{itemize} 

We emphasize that these lists are incomplete. The ownership and schedule (milestones) for the complete
lists of deliverables are managed by professional management
tools (the RFL effort is ``projectized''). They are tracked and their progress discussed in supporting documents. 


\subsection{Decision making and approval process} 

The SFLcg is empowered to make all the decisions required for efficient execution of the RFL project.
They are encouraged to take broad input from various stakeholders and keep the RCOC informed about
the progress. 

An approval process of RFL products that is efficient but complete will allow all products to be created in a
timely fashion. All written and visual RFL communications products will be approved following the standard
process established for the momentum-building background science press releases. 


\subsection{Embargo policies and responsibilities \label{embargo}}

To ensure the confidentiality and integrity of sensitive information within our collaboration prior to the official
public RFL release, the following embargo policy, derived from the Rubin document ``Information Sharing during
Commissioning'' by Keith Bechtol \& Steve Ritz (2024; https://sitcomtn-076.lsst.io/), is in effect. 

All Rubin, AURA and SLAC staff, as well as Rubin community members working on observations, data management,
EPO, communications etc. shall adhere to the following rules of confidentiality. 

All specific information about RFL imaging products is deemed confidential until the embargo expires at the time
of the RFL press release. Specific information about RFL images and targets can only be shared with the people
involved in the RFL campaign. Nothing specific about RFL shall be shared outside, including on social media.
General information about RFL can be shared with the community, for instance at Rubin Community Workshops,
e.g. ``The System First Light observations will be taken over a 3-week period currently slated to start in March 2025.
Several targets will be observed depending on their visibility at the time of observation, and their suitability to
demonstrate various aspects of Rubin Observatory. Several committees consisting of experts are working on different
aspects of Rubin First Look event, in order to maximize the press and social media visibility.''

The Rubin Construction Director will maintain the list of personnel with access. A subset of staff working on the
press release images will be the only ones who are granted access to image products. 
Access to embargoed information is limited to essential personnel only. Secure methods for sharing information
will be applied (e.g., encrypted emails, access controlled documents etc). All embargoed documents and images
will be marked with ``CONFIDENTIAL''. 

Embargoed information shall not be printed unless absolutely necessary. Any printed materials shall be stored
in locked and secure locations. Printed materials should be disposed via secure shredding methods after the
embargo period.

All media inquiries should be directed to the designated media liaisons. No staff or community members shall
provide comments or information about FL to colleagues or to the media before the embargo lift date.

All suspected or actual breaches of this policy should be immediately reported to Rubin Construction Director.
Violations of this policy may result in disciplinary actions, including termination of access to the RFL project.

The Rubin Construction Director is responsible for enforcing this policy and ensuring compliance.

Exceptions to this policy can only be granted by the Rubin Construction Director in writing and after informing
the SFLcg. 

  