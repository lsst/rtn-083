\section{The strategy for media engagement} \label{sec:strategy}
\begin{itemize}
\item - momentum building prior to RFL event
\item - media engagement prior to RFL event
\item - press conference
\item - social networks
\item - interviews
\item - web presence
\item - supporting hub events
\item - metrics for measuring success
\end{itemize}

\subsection{First Light Communication Channels}
Channels under our control:

\begin{itemize}
\item Expert interviews
\item Media visits
\item Press releases
\item AURA/Rubin/NOIRLab/NSF/SLAC/DOE social media incl. for live streaming of the main press conference
\item AURA/Rubin/NOIRLab/NSF/SLAC/DOE websites
\item Auxiliary channels such as Reddit, Ted Talks etc.
\end{itemize}

\emph{External} channels:
\begin{itemize}
\item Print Media - general public - newspapers, websites, etc.
\item Broadcast media
\item Social media
\item Scientific trade publications (Sky \& Telescope, Scientific American etc.)
\end{itemize}


\subsection{Momentum-building}
Momentum building activities are important to build a following of people interested in the Rubin Observatory before the first light images are released. These activities also include educating the media and the public about the scientific goals of Rubin and its amazing technical achievements.

Some momentum building activities include:
\begin{enumerate}
\item Press releases
\begin{enumerate}
\item Topical press releases focusing on the various science areas and featuring prominent Rubin scientists
\item Organizational press releases focusing on construction milestones
\end{enumerate}
\item AAS 2025
\begin{enumerate}
\item Exhibits and interviews at winter AAS January 2025
\item Exhibits and interviews at summer AAS June 2025
\item General ramp-up of (reactive) ongoing in-person Media visits to the summit.
\end{enumerate}
\end{enumerate}



