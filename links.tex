\section{Links to detailed implementation documents} \label{sec:links}

\begin{itemize}
\item - detailed projectized deliverable list and schedule for each WG
\item - (embargoed) SFL data taking and data processing plans
\item - press kit contents (including Rubin background info and main science drivers)
\item - press conference organization
\item - media engagement details
\item - social networks strategy
\item - website and other support on the day of RFL event
\item - etc...
\end{itemize}


\subsection{Spokespeople Preparation}
Document: \href{https://docs.google.com/document/d/14pbmwzvYFVolCZJV7rvoWdZQHR82a8kKWZrBjAUKzzI/edit}{Rubin FL Media Training }

It is essential to provide media training for all people who will be identified for interviews. Even if they have previous experience with the media, this training will inform them of Rubin specific messaging and topics to avoid.

\begin{enumerate}
\item Media training for Rubin spokespeople
\item Rubin slide deck for spokespeople and general members of the scientific community
\item Main messages and talking points
\end{enumerate}

\subsection{Social Media Campaign}
Document: \href{https://docs.google.com/document/d/1-STd7aO3-_T73zZEHpxTrj2OtYoc_e6QERagPVNbfFQ/edit#heading=h.rkin5k30kjf}{Social Media Strategy for First Light}

In the lead up to the FL image release it is important to build and engage with the social media audience. Many of our stakeholders are active in social media in addition to consuming traditional and new media. A campaign that tells the story of Rubin and the science it will do will get our audience excited for the FL images. A brainstorming session to create the strategy and story will be planned. Some elements include:
\begin{enumerate}
\item Strategy for lead up to FL press release
\item Images
\item Text - alt and main
\item Graphics
\item Videos
\item Outreach to influencers
\end{enumerate}

\subsection{Media Engagement}
Engaging with the media early and often will help ensure a successful First Light images release. The goal is to make it easy for the media to gain access to “behind the scenes” information that they can use in their stories. Virtual tours and Q\&A panels are a way to include those members of the media who can’t travel to the observatory in Chile. These events will also build relationships between the communications team and the media for future coverage of the observatory.

The process can be similar to NASA’s media engagement. A call to the media can be sent out via email and on social media one week before the event. Those who want to attend the tour via zoom or telephone and ask questions will need to register in advance. Those who don’t register can still view the tour via YouTube or FaceBook.

Some possible events include:
\begin{enumerate}
\item Virtual media tours
\item Social media days
\item Proactive in-person group media visits to the summit arranged in preselected slots with a curated experience
\item Virtual panels for Rubin Q\&A
\begin{enumerate}
\item Who: Rubin user/scientist, Rubin engineer, Rubin comms person
\item Why: opportunity to ask general questions
\end{enumerate}
\end{enumerate}

\subsection{General Press Kit about Rubin Observatory}
Document: \href{https://docs.google.com/document/d/1aFM2bdE-ogtjh63nyE-UTsOtcQC9Ybis1s3LjQipL1Y/edit}{Rubin Media Kit Design Doc}

A Rubin Press Kit is essential and is required ASAP. Its purpose is to inform the media about Rubin Observatory and provide them with all the material they need to easily write stories and social media posts about Rubin, and ultimately about the FL images. The goal of the Press Kit is to make the media’s job covering Rubin as easy as possible.

Many publications no longer have dedicated science writers, so the information in the Rubin Press Kit must be ready for them to use “as is” and should be written for a non-science audience. That means the text should be free of jargon, and all science concepts must be explained. In addition, readily understandable comparisons or equivalencies are strongly recommended for the media who will be creating content for general audiences (e.g., one cell on the Sun is the size of Texas; one Rubin image is the size of 45 full moons).

The Press Kit will be a single document in PDF format.

\subsection{First Light Press Package Components}

The press package for the FL image release must be complete and easily accessible by media. The press package, along with the press kit, should provide the media covering Rubin First Light with everything they need for both traditional and social media products.

NOTE — all web servers must be prepared to accommodate high traffic.

First Light Press Package Contents
The contents of the first light press release package is listed below. Also a mock-up can be seen in \url{https://docs.google.com/presentation/d/1g1XEtyx2_pgcLC2sSWYDe6memaTYsjEaLmI6X6DzfiU/edit#slide=id.p}{this slide deck}.

\begin{enumerate}
\item Press release copy text
\item Images (in multiple formats)
\item Videos (in multiple formats, incl. fulldome)
\begin{enumerate}
\item Zooms
\item Pans
\item Dynamic videos (for instance RR Lyrae variables\footnote{Example: \url{https://www.eso.org/public/videos/eso1636a/}})
\item Video explainer, e.g.
\begin{enumerate}
\item \#SLAC-Explains
\item CosmoView
\item Reels
\end{enumerate}
\end{enumerate}
\item Image captions
\item Alt text
\item Video captions
\item Curated social-media-ready assets (graphics prepared for each platform, hashtags, etc)
\item Background photos and videos.
\begin{enumerate}
\item Three archives of high-end background images and videos have been set up
\item Also, it will be planned to have photographers on site to document the weeks of the FL observations.
\end{enumerate}
\item Contact list of media-trained cadre of scientists and engineers for interviews on specific topics
\item Links to
\begin{enumerate}
\item Background image and video galleries (well organized with multiple resolutions) with explainer videos, scientist interviews, SLAC camera videos etc.:
\begin{enumerate}
\item Rubinobs.org (\url{https://rubin.canto.com/v/gallery/album/HDSNU?display=curatedView&viewIndex=2})
\item \href{https://noirlab.edu/public/images/archive/category/rubin/}{NOIRLab.org} (\href{https://noirlab.edu/public/images/archive/category/rubin/}{image archive}, \href{https://noirlab.edu/public/videos/archive/category/rubin/}{video archive})
\item \href{http://NOIRLab.org}{Slac}
\end{enumerate}
\item “About Rubin” press kit PDF (see above)
\item Sonification product
\item Data in the Rubin science archive
\end{enumerate}
\end{enumerate}

\subsection{First Light Embargo policy}
Following the Information Sharing during Commissioning \cite{sitcomtn-076}.

To ensure the confidentiality and integrity of sensitive information within our collaboration prior to the official public FL release, the following embargo policy is in effect.

All AURA and SLAC staff as well as Rubin community members working on observations, data management, EPO, communications etc. shall adhere to the following rules of confidentiality.

All specific information about FL imaging products is deemed confidential until the embargo expires at the time of the FL press release. Specific information about FL images and targets can only be shared with the people involved in the FL campaign. Nothing specific about FL shall be shared outside, including on social media. General information about FL can be shared with the community, for instance at PCW. E.g. “The First Light observations will be taken over a 3-week period currently slated to start x March 2025. Several targets will be observed depending on their visibility at the time of observation, and their suitability to demonstrate various aspects of Rubin Observatory. Several committees consisting of experts are working on different aspects of First Light, in order to maximize the press and social media visibility.”

The Rubin Construction Director will maintain the list of personnel with access. A subset of staff working on the press release images will be the only ones who are granted access to image products.

Access to embargoed information is limited to essential personnel only. Secure methods for sharing information will be applied (e.g., encrypted emails, access controlled documents etc). All embargoed documents and images will be marked with "CONFIDENTIAL".

Embargoed information shall not be printed unless absolutely necessary. Any printed materials shall be stored in locked and secure locations. Printed materials should be disposed via secure shredding methods after the embargo period.

All media inquiries should be directed to the designated media liaisons. No staff or community members shall provide comments or information about FL to colleagues or to the media before the embargo lift date.

All suspected or actual breaches of this policy should be immediately reported to Rubin Construction Director. Violations of this policy may result in disciplinary actions, including termination of access to the FL project.

The Rubin Construction Director is responsible for enforcing this policy and ensuring compliance.

Exceptions to this policy can only be granted by the Rubin Construction Director in writing and after informing the SFLcg.
