\section{Context, Goals, Stakeholders and Audiences}

The term First Look (sometimes First Light) is usually used to signify the release of the first publicly released images from a new telescope (or a new instrument) constructed from calibrated astronomical data after it has been constructed. This is typically not the engineering tests performed to adjust the complex systems, but the first images that pass the requirements and fully illustrate the actual scientific potential of the new system. For Rubin, First Look is an artificial ‘event’ which is just the chosen day of the press release. It is coupled with the construction progress and the progress of the first calibrated observations, but once these are done the detailed date of this event is driven by the progress of the composition and cleaning of color images.


Rubin’s First Look (hereafter RFL), preceeded by internal System First Light milestone, will be a pivotal event in the life of the observatory, marking its transition from a technological project to a tool for exploring the Universe. It will demonstrate scientific readiness: confirm that the telescope is operational and capable of capturing scientific data. And it will inspire confidence: assure the scientific community and funding bodies that the project is on track and ready for scientific research.

Strategic planning is a necessity when endeavoring to engage with the media and break through the daily newscycle to reach the broader public. It is imperative not only to provide materials and trained subject matter experts for our media partners, but to also develop relationships long before the actual image release. Success in these activities requires a plan that starts early enough and builds towards the image release in 2025. 

This strategic plan outlines both of the elements required: the {\it products} that need to be created for media and
the {\it strategy for media engagement}. Estimated timelines have been developed as time relative to RFL as the final RFL date
is still shifting. RFL public announcement is anticipated for the first half of 2025. 

Later media engagements, and naming and dedication ceremonies in Chile, are planned separately. 
 

\subsection{Rubin's mission statement and adjustments for intended audiences}


Major advances in our understanding of the universe have historically arisen from dramatic improvements in our ability to ``see''.
Until recently, most astronomical investigations have focused on small samples of cosmic sources or individual objects. This is because
even the largest telescope facilities typically had rather small fields of view, and those with large fields of view could not detect very faint sources. With all of our existing telescope facilities, we have still surveyed only a small fraction of the observable universe (except when considering the most luminous quasars).

However, recent advances in technology (in particular, telescopes, sensors and computers) have made it possible to move beyond the traditional observational paradigm and to undertake large-scale sky surveys. 
Rubin Observatory was specifically designed and constructed to deliver the most ambitious optical sky survey to date:  the Legacy Survey
of Space and Time (LSST). The LSST design is driven by four main and broad science themes: probing dark energy and dark matter, taking an inventory of the solar system, exploring the transient optical sky, and mapping the Milky Way. The most unique feature of LSST, compared
to previous sky surveys, is an enormous number of detected objects: about 20 billion galaxies and a similar number of stars will be detected -- for the first time in history, the number of cataloged celestial objects will exceed the number of living people! 

The key aspect of LSST is that all scientific investigations will utilize a common image and catalog databases constructed from an optimized observing program. This program will be not be subject to common Time Allocation Committee but instead will be executed using an essentially autonomous AI program (the so-called LSST Scheduler). LSST data will enable an extremely broad range of new scientific investigations. A particular aspect of Rubin Observatory and LSST is an exquisite software suite that will both run the Observatory
and process images (including object finding and measurement of properties such as position, brightness, shape, colors) to science-ready form. The resulting science-ready data products will be distributed to many thousands of scientists around the world, with time-sensitive
information about objects that change in brightness or position available within an unprecedentedly short period (60 seconds) after
acquiring image. 

In addition, the worldwide attention received by outreach tools such as Sky in Google Earth and the World Wide Telescope, and the hundreds of thousands of volunteers classifying galaxies in the Galaxy Zoo project and its extensions, demonstrate that the impact of sky surveys extends far beyond fundamental science progress and reaches all of society. For this reason, from the very start, the Rubin Observatory Construction project included an Education and Public Outreach (EPO) component which is anticipated to have high impact with the interested public, planetariums and science centers, citizen science projects, as well as middle school through university educational programs. 
Since its official completion, EPO is now a functional unit of Rubin Operations.

These aspects of Rubin Observatory and LSST led to the following Rubin Mission Statement {\bf ``Rubin Observatory’s mission is to build a well-understood system that will produce an unprecedented astronomical data set for studies of the deep and dynamic universe, make the data widely accessible to a diverse community of scientists, and engage the public to explore the Universe with us.''}

While this mission statement succintly captures all of the primary aspects of Rubin Observatory and LSST, it is not adequate
for reaching out to the broadest possible audience (see below for a discussion of intended audiences). A set of key high-level
messages designed for broadest possible audiences are discussed in Section~\ref{RFLproducts}.



\subsection{RFL goals and high-level strategy for achieving them}

The RFL images, and accompanying messages, will be designed to make a huge splash, with broad coverage. 
They will show capability and promise for early science in a manner that deeply engages the public and encourages
interest for subsequent early science releases. They will connect familiar phenomena with the unfamiliar Rubin capabilities and create a “wow!”-effect.

The goals of the FL campaign are listed here in order of priority:
\begin{enumerate}
\item Capture a high level of media attention (among mainstream media including Chilean media, as well as astronomy-focused media)
\begin{itemize}
  \item Objective \#1: Rubin images are featured ``above the fold'' in a major US newspaper 
  \item Objective \#2: Rubin images are featured ``above the fold'' in a major Chilean newspaper
  \item Objective \#3: At least one viral post on social networks 
\end{itemize}
\item Demonstrate Rubin's unique, substantial, and awe-inspiring science potential to the world
\begin{itemize}
  \item Objective \#1: Your neighbors have heard of Rubin Observatory the day after release of first light images
  \item Objective \#2: Media stories are still regularly being written about Rubin 6 months after release of first light images
\end{itemize}
\item Acknowledge funding organizations whose contributions made Rubin Observatory a reality
\begin{itemize}
  \item Objective \#1: Funding agencies sufficiently acknowledged for their support within FL media products
  \item Objective \#2: Funding agencies credited by media
\end{itemize}
\end{enumerate} 


\subsection{RFL stakeholders}

Key stakeholders include NSF, DOE, AURA, SLAC, NOIRLab, LSST DA, and AURA-O.

\subsection{Intended audiences}

The RFL audiences are listed here in order of priority:
\begin{enumerate}
\item News media, national and international -- engaged through communications products
\item Funding stakeholders and decision makers from the Congress and federal agencies (NSF and DOE)
\item Broad public, the science and technology-attentive public -- engaged through media, websites, social media
\item Potential Rubin users and citizen scientists, educators
\end{enumerate}


\subsection{Definitions of closely related Rubin events}

There are several Rubin milestones/events that are similar, or sound similar, so these brief definitions
are intended to avoid confusion:
\begin{itemize}
\item {\it Rubin First Photon:} The first on-sky image obtained with LSSTCam on the Simony Survey Telescope
  It is a well-defined in time and anticipated in January/February of 2025. This first image is not expected to be
  of scientific quality but it will mark the start of the full-system commissioning and operational optimization.
\item {\it System First Light:} It is defined by image quality, with emphasis on delivered seeing.
Nominally expected about 2 months after Rubin First Photon. It will be declared days if not weeks after
the actual data were taken. Decision to declare it may be coupled with an end-to-end data processing effort
and its success. Decision to declare it will be made by the Rubin Construction Directorate and Project Management
Office, as recommended by the Project Science Team.  
\item {\it Rubin First Look}:  Public release of System First Light data products (and supporting media products
  and events, culminating with a  Press Conference). This is the event discussed in this document.
  It is nominally expected about 3 weeks after System First Light data were declared.
\end{itemize}
  



