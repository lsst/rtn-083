\section{Goals, design details and list of RFL products} \label{sec:goals}

\begin{itemize}
\item  - key high-level messages
\item - images and image-based products (only non-embargoed details)
\item - additional products for varying audiences (e.g. self-contained press kit for media)
\end{itemize}

The FL images, and accompanying messages are designed to make a huge splash, with broad coverage.

They will show capability and promise for early science in a manner that deeply engages the public and encourages interest for subsequent early science releases.

They will connect familiar phenomena with the unfamiliar Rubin capabilities and create a \emph{wow!}-effect.

The goals of the FL campaign are listed here in order of priority\footnote{ These are a focused subset of the RCOC goals focusing on the FL media release. }.


\begin{enumerate}
\item Capture a high level of media attention (among mainstream media including Chilean media, as well as astronomy-focused media)
\begin{enumerate}
\item Objective \#1: Rubin images are featured “above the fold” in a major US newspaper
\item Objective \#2: Rubin images are featured “above the fold” in a major Chilean newspaper
\item Objective \#3: At least one viral post (viral = more than XXX reposts)
\end{enumerate}
\item Demonstrate Rubin's unique, substantial, and awe-inspiring science potential to the world
\begin{enumerate}
\item Objective \#1: Bob’s parents’ neighbors have heard of Rubin Observatory the day after release of first light images
\item Objective \#2: Media stories are still regularly being written about Rubin 6 months after release of first light images
\end{enumerate}
\item Acknowledge funding organizations whose contributions made Rubin Observatory a reality
\begin{enumerate}
\item Objective \#1: Funding agencies sufficiently acknowledged for their support within FL media products
\item Objective \#2: Funding agencies credited by media
\end{enumerate}

\end{enumerate}


\subsection{Main Messages about First Light and First Images }
The following messages are about FL only, not Rubin in general.

The messages will adhere to the following principles:
\begin{enumerate}
\item Messages should focus on why this should be above the NYT fold
\item No jargon (like “system”)
\item Messages should make things as easy as possible for media (sound bites)
\item Include “familiar equivalences” that people can understand (e.g., DKIST sees solar structure the size of Texas and even the island of Manhattan)
\item Use a single name to refer to the observatory, camera and survey. Initial: NSF–DOE Vera C. Rubin Observatory thereafter Rubin.
\item Refer to other names (LSST Camera, Simonyi Survey Telescope) ONLY when necessary, preferably only in the “about” section after the main release).
\end{enumerate}

Primary Messages about First Images (no more than 3 maximum; short and punchy)
\begin{itemize}
\item This image demonstrates a new way of studying the sky…
\item Key science from this image will be ….
\item Today’s images are just the beginning. This is a first look  …
\end{itemize}

Secondary Messages about First Images and Rubin more generally
\begin{itemize}
\item Rubin Observatory is a major feat of engineering
\item Rubin Observatory uses innovative optics
\item Rubin Observatory named after Vera C. Rubin.
\item Strong role for citizen science …
\item Educational materials are ready and available in both English and Spanish
\item Built by a large collection of people
\item Public-private partnership
\end{itemize}

