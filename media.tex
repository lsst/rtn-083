\section{The Media Engagement Strategy}


\subsection{RFL Communication Channels}

The RFL impact will be maximized with maximized number of communication channels.
The channels under our control include
\begin{itemize}
\item Press releases
\item Expert interviews
\item Media visits 
\item AURA/Rubin/NOIRLab/NSF/SLAC/DOE social media incl. for live streaming of the main press conference
\item AURA/Rubin/NOIRLab/NSF/SLAC/DOE websites 
\item Auxiliary channels such as Reddit, Quora, and Ted Talks. 
\end{itemize} 

``External'' channels that not under our control include
\begin{itemize}
\item Print Media (for general public: newspapers, websites, etc.)
\item Broadcast media
\item Social media
\item Scientific trade publications (Sky \& Telescope, Scientific American etc.)
\end{itemize} 

All communication products will be created in both English and Spanish. 

Communication products are needed for, or build towards, the images and press release for the FL images.
They fall into some main categories listed below. Each of the deliverables will have their own more detailed planning documents. 


\subsection{Momentum building prior to RFL event}

Momentum building activities are important to build a following of people interested in the Rubin Observatory before the first light images are released. These activities also include educating the media and the public about the scientific goals of Rubin and its amazing technical achievements. 

Some momentum building activities include:
\begin{itemize} 
\item Organizational press releases focusing on construction milestones
\item Topical press releases focusing on the various science areas and featuring prominent Rubin scientists
\item Exhibits and interviews at winter AAS January 2025
\item Exhibits and interviews at summer AAS June 2025
\item General ramp-up of (reactive) ongoing in-person Media visits to the summit.
\end{itemize}


\subsection{Media engagement prior to RFL event}

Engaging with the media early and often will help ensure a successful RFL. The goal is to make it easy for the media
to gain access to ``behind the scenes'' information that they can use in their stories. Virtual tours and Q\&A panels
are a way to include those members of the media who cannot travel to the observatory in Chile. These events will
also build relationships between the communications team and the media for future coverage of the observatory. 

The process can be similar to NASA’s media engagement. A call to the media can be sent out via email and on social
media one week before the event. Those who want to attend the tour via zoom or telephone and ask questions will
need to register in advance. Those who do not register can still view the tour via, e.g., Rubin's YouTube channel. 

Some possible events include:
\begin{itemize} 
\item Virtual media tours
\item Social media days
\item Proactive in-person group media visits to the summit arranged in preselected slots with a curated experience
\item Virtual panels for Rubin Q\&A (Who: Rubin user/scientist, Rubin engineer, Rubin comms person; Why: opportunity
               to ask general questions)
\end{itemize} 



\subsection{Press conference}

NSF and DOE have informed Rubin team that NSF is interested and capable of running the primary press
conference for RFL  and definitely envision it taking place either at the NSF HQ or at the National Press Club
(where they have the infrastructure, experience and staff to do such a conference to the right protocols and
with the expected remote and in-person engagement of press).

{\bf Press conference in D.C. will be the primary RFL media event.}
The RFL team should engage NSF and DOE in discussion to position the
formal press conference with parallel and associated activities.

Media events, especially embargoed ones, will help get the attention of the press for the RFL event.
Embargoes allow the press to prepare their stories ahead of the actual release date so when the images
are released the media stories are also ready to go. NSF has a press list of trusted media contacts that
can be the starting point for the invites to the embargoed press release. 

The key RFL media engagement steps will be executed in this order:  
\begin{enumerate}
\item Outreach to key media contacts
\item Embargoed “Get Ready for Rubin” Press conference: ``Background'' virtual press conference held in
conjunction with NSF, DOE, following: 
  \begin{itemize} 
    \item Embargoed release without images distributed one day prior
    \item Invite for this virtual press conference sent to trusted journalists
    \item Panel of experts will answer press questions
  \end{itemize}
\item Images released to embargo group 2 hours before general release
\item Primary Priority: Main in-person press conference day of image release
  \begin{itemize}
  \item  Hosted jointly by NSF and DOE in Washington DC (work with NSF to secure the National Press Club,
        which was the site of the Event Horizon Telescope and Gravitational Wave Optical Follow-up events)
  \item Invite to go out to embargo and non-embargo press and via social media
  \item Aim for noon Eastern to enable supporting events in Europe and further west, to Pacific time zone (see below) 
  \end{itemize}
\item Secondary Priority: Concurrent press conference in Santiago, in person and streamed from Primary site 
\item Tertiary Priority: press event could be streamed live to various other venues (see below)
\item Post-RFL follow-up media interviews
\end{enumerate} 

The ‘point-of-no-return’ RFL moment will be the date of the Media advisory announcement.
The decision about executing this step will undergo a review (SFL Media Alert Gate review).
%Criteria:
 


\subsection{Supporting hub events}

Given the exquisite preparations for the RFL and extensive resulting media products, it will be
relatively easily to organize an event concurrent with the press conference in D.C. The essential
driver is to use the main event to trigger and partially support a local event, which would also
showcase local Rubin aspects (construction and operations work, Science Collaborations, local
EPO intersests, etc.). The main press release text could be easily modified to allow room for
specific local content (including quotes) and promote local organizations. 

A number of groups have already expressed interest (a number of hubs in US, several hubs in
UK and France, also hubs from Italy, Hungary, Slovenia, Serbia, Croatia). The team should keep
them engaged and informed about progress, and send Press Kit and detailed timing for the
main press conference to them as soon as they are available. 



\subsection{Social networks}

In the lead up to the RFL image release it is important to build and engage with the social media audiences.
Many of our stakeholders are active in social media in addition to consuming traditional and new media.
A campaign that tells the story of Rubin and the science it will do will get our audience excited for the RFL images.

The team should have a brainstorming session to create the strategy and stories to tell. Some elements include:
\begin{itemize}
\item Strategy for lead up to RFL press release
\item Images
\item Text  (alt and main)
\item Graphics
\item Videos
\item Outreach to influencers
\end{itemize}

We note that NSF and DOE plan to amplify Rubin stories on their social media feeds. 

\subsection{Interviews and Spokespeople Preparation \label{interviews}} 

Pre-recorded short video interviews with selected Rubin team members and expert scientists will be included
in the Press Kit.

It is essential to provide media training for all people who will be identified for interviews. Even if they have
previous experience with the media, this training will inform them of Rubin specific messaging and topics to avoid.
After people for interviews have been selected, the team will provide to them
\begin{itemize}
\item Media training
\item Rubin slide deck (both for spokespeople and general members of the scientific community)
\item Main messages and talking points (consistent with key high-level messages, see Section~\ref{khlm})
\end{itemize} 



%\subsection{Web presence}


\subsection{Metrics for measuring success and their aspirational goals}

The following metrics have been used before, both by Rubin and other teams, to measure
the success of press releases and other media campaigns:
\begin{itemize}
\item Front page(s) of major newspaper(s) in US and Chile (goal: at least one in each country)
\item Near the top of trending on social media in the US and Chile (goal: at least in the top 5)
\item Meltwater’s ``number of theoretical readership'' metric for all articles covering the RFL
               should be in billions (over 4 billion to be comparable to the EHT metric).
\item Number of visits to main Rubin-related websites should increase by at least 50\%
\item Number of visits to Rubin's YouTube channel should increase by at least 50\%
\item A famous non-scientist talks about Rubin (e.g., President, ISS Astronaut, major influencer, etc.)
\item Several memes get significant attention on social media
\item A measureable impact on Google search trend
\item Mentioned in major TV news
\item Rubin is the google search image for day
\end{itemize}

Note that these goals, though based on prior experience, are aspirational since most are
beyond our direct control. The most meaningful metrics and measures of success will be
those that can be compared to other similar events (e.g. Event Horizon Telescope announcement
in 2019, see https://www.capjournal.org/issues/26/26\_11.pdf). 



  