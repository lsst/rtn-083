\documentclass[LPM,lsstdraft,authoryear,toc]{lsstdoc}
%\documentclass[OPS,authoryear,toc]{lsstdoc}
\input{meta}

% Package imports go here.

% Local commands go here.

%If you want glossaries
%\input{aglossary.tex}
%\makeglossaries

\title{Rubin First Look Public Announcement Strategy}

% This can write metadata into the PDF.
% Update keywords and author information as necessary.
\hypersetup{
    pdftitle={Rubin First Look Public Announcement Strategy},
    pdfauthor={William O'Mullane},
    pdfkeywords={}
}

% Optional subtitle
% \setDocSubtitle{A subtitle}

\author{%
Luz Mar\'{i}a Aguirre, Keith Bechtol, Bob Blum, Lars Lindberg Christensen, Ranpal Gill, Heidi Hammel,
\v{Z}eljko Ivezi\'{c}, Victor Krabbendam, Shari Lifson, Kristen Metzger, William O'Mullane, Steve Ritz,
Aaron Roodman, Sandrine Thomas, Alan Strauss, Alejandra Voigt, Beth Willman (incomplete?)
}

\setDocRef{RTN-083}
\setDocUpstreamLocation{\url{https://github.com/lsst/rtn-083}}

\date{\vcsDate}

% Optional: name of the document's curator
% \setDocCurator{The Curator of this Document}

\setDocAbstract{%
This strategic plan is focused solely on the activities and products
needed to produce significant media impact for the first image release
(Rubin First Look) from the newly completed Vera C. Rubin Observatory.
Later media engagements, and naming and dedication ceremonies in
Chile, are planned separately. This document summarizes strategic
concepts and guidance while implementation details are described in supporting documents.
Rubin First Look public announcement is
anticipated for the first half of 2025. \\

{\bf Note: while this document is written in future tense, most
  of described deliverables are in progress or have been already completed.}
}

% Change history defined here.
% Order: oldest first.
% Fields: VERSION, DATE, DESCRIPTION, OWNER NAME.
% See LPM-51 for version number policy.
\setDocChangeRecord{%
  \addtohist{1}{2024-07-24}{Unreleased, first port from google doc draft.}{\v{Z}eljko Ivezi\'{c}}
}


\begin{document}

% Create the title page.
\maketitle
% Frequently for a technote we do not want a title page  uncomment this to remove the title page and changelog.
% use \mkshorttitle to remove the extra pages

% ADD CONTENT HERE
% You can also use the \input command to include several content files.

\section{Goals, design details and list of RFL products} \label{sec:goals}

\begin{itemize}
\item  - key high-level messages
\item - images and image-based products (only non-embargoed details)
\item - additional products for varying audiences (e.g. self-contained press kit for media)
\end{itemize}

The FL images, and accompanying messages are designed to make a huge splash, with broad coverage.

They will show capability and promise for early science in a manner that deeply engages the public and encourages interest for subsequent early science releases.

They will connect familiar phenomena with the unfamiliar Rubin capabilities and create a \emph{wow!}-effect.

The goals of the FL campaign are listed here in order of priority\footnote{ These are a focused subset of the RCOC goals focusing on the FL media release. }.


\begin{enumerate}
\item Capture a high level of media attention (among mainstream media including Chilean media, as well as astronomy-focused media)
\begin{enumerate}
\item Objective \#1: Rubin images are featured “above the fold” in a major US newspaper
\item Objective \#2: Rubin images are featured “above the fold” in a major Chilean newspaper
\item Objective \#3: At least one viral post (viral = more than XXX reposts)
\end{enumerate}
\item Demonstrate Rubin's unique, substantial, and awe-inspiring science potential to the world
\begin{enumerate}
\item Objective \#1: Bob’s parents’ neighbors have heard of Rubin Observatory the day after release of first light images
\item Objective \#2: Media stories are still regularly being written about Rubin 6 months after release of first light images
\end{enumerate}
\item Acknowledge funding organizations whose contributions made Rubin Observatory a reality
\begin{enumerate}
\item Objective \#1: Funding agencies sufficiently acknowledged for their support within FL media products
\item Objective \#2: Funding agencies credited by media
\end{enumerate}

\end{enumerate}


\subsection{Main Messages about First Light and First Images }
The following messages are about FL only, not Rubin in general.

The messages will adhere to the following principles:
\begin{enumerate}
\item Messages should focus on why this should be above the NYT fold
\item No jargon (like “system”)
\item Messages should make things as easy as possible for media (sound bites)
\item Include “familiar equivalences” that people can understand (e.g., DKIST sees solar structure the size of Texas and even the island of Manhattan)
\item Use a single name to refer to the observatory, camera and survey. Initial: NSF–DOE Vera C. Rubin Observatory thereafter Rubin.
\item Refer to other names (LSST Camera, Simonyi Survey Telescope) ONLY when necessary, preferably only in the “about” section after the main release).
\end{enumerate}

Primary Messages about First Images (no more than 3 maximum; short and punchy)
\begin{itemize}
\item This image demonstrates a new way of studying the sky…
\item Key science from this image will be ….
\item Today’s images are just the beginning. This is a first look  …
\end{itemize}

Secondary Messages about First Images and Rubin more generally
\begin{itemize}
\item Rubin Observatory is a major feat of engineering
\item Rubin Observatory uses innovative optics
\item Rubin Observatory named after Vera C. Rubin.
\item Strong role for citizen science …
\item Educational materials are ready and available in both English and Spanish
\item Built by a large collection of people
\item Public-private partnership
\end{itemize}


\section{Design Guidance and List of RFL Products \label{RFLproducts}}

\subsection{Key high-level messages \label{khlm}}

Key high-level messages will capture the essence of Rubin Observatory and LSST, and will be
consistent with Rubin's mission statement discussed above. Furthermore, these messages
will adhere to the following principles:
\begin{itemize}
\item Messages should focus on, e.g., why this news should be ``above the NYT fold''
\item There should be no jargon (like ``system'')
\item Messages should make things as understandable as possible for media (sound bites)
\item Include ``familiar equivalences'' that people can understand (e.g., DKIST sees solar structure the size of Texas;
     in Rubin's case, such equivalences and other analogies will be crucial to capture the size and complexity of its dataset) 
\item Use a single name to refer to the observatory, camera and survey. Initial: NSF–DOE Vera C. Rubin Observatory thereafter Rubin. 
\item Refer to other names (LSST Camera, Simonyi Survey Telescope) ONLY when necessary, preferably only in the ``about'' section after the main release).
\end{itemize}

Key high-level messages should be organized in
\begin{enumerate}
\item Primary messages about Rubin First Look
\item Secondary messages about Rubin First Look and more generally about Rubin and LSST
\end{enumerate}

These key high-level messages should be completed (apart from embargoed details about
the actual RFL dataset) well before the RFL event. 
 

\subsection{Images and image-based products}

The most successful past image releases from major telescopes were not only stunning visually, but also told a story scientifically. The team developing this plan and those executing it should remember that it is the scientific potential that captures the media and public attention. The RFL release is not intended to showcase actual publishable science results, but provide the most evocative images possible to illustrate Rubin’s wide-field, high cadence and amazing science potential. Using such proxies is also important as so-called Science press releases will need to be based on peer-reviewed science papers.


There are huge expectations from Rubin stakeholders for this event -- most people are envisioning a splash similar to what JWST and Event Horizon Telescope achieved. Yes, we need to acknowledge that Rubin’s images will not be even remotely as spectacular as those from
JWST (or even from DKIST) because of space-based resolution vs. ground-based resolution difference. Indeed, Rubin Observatory is not designed to produce spectacular images; its purpose is to produce {\bf a very large number of very large 
images} with ground-based (mediocre, by space standards) resolution. In some sense, an ultimate celebratory moment 
will be something like ``LSST’s last photon'' in 2035, rather than first photons and First Look in 2025 because
{\bf it is the final LSST dataset that will be unprecedented and impressive but not Rubin's first light images}. 

Given that fact, the best strategy to attract and excite media rests on two important pillars:
\begin{enumerate} 
\item Do not rely solely on static images but use tools for zoom in/out to convey simultaneously large field of view and many pixels 
\item Showcase time domain aspects and unprecedented Rubin's etendue (about 100 times faster surveying speed than other 8m telescopes) 
\item As an extend goal, introduce Rubin software and AI-powered data interpretation (``astronomers cannot look at so many images, but computers can!'')
\end{enumerate}

The Images Working Group (see Section~\ref{WGs}) is charged with designing and executing an observing
program that will follow this strategic guidance and produce a dataset and data products that will
maximize the success of RFL. 

Note that this discussion intentionally avoids embargoed details (see Section~\ref{embargo}). 


\subsection{Press kit}

In addition to the actual RFL image-based products, an informative, self-contained, detailed and easy to use Press Kit
will be a crucial product to ensure RFL success. Its purpose is to inform the media about Rubin Observatory and provide them with all the material they need to easily write stories and social media posts about Rubin, and ultimately about the RFL images.
In other words, the Press Kit needs to make the media’s job covering Rubin as easy as possible. 
The Press Kit will be created in both English and Spanish.

Many publications no longer have dedicated science writers, so the information in the Rubin Press Kit must be ready
for them to use ``as is'' and should be written for a non-science audience. That means the text should be free of jargon,
and all science concepts must be explained in non-specialist language. In addition, readily understandable comparisons,
analgoies and equivalencies are strongly recommended for the media who will be creating content for general audiences
(e.g., one Rubin image is the size of 45 full moons).

The Press Kit will include: 
\begin{itemize}
\item Press Release text (including embargo information) 
\item Information about and pointers to image-based products and tools, discussed above.
\item Pre-recorded short video interviews with selected Rubin team members and expert scientists (see Section~\ref{interviews}) 
\item Rubin textual background information (see below).
\item Videos about Rubin Observatory and LSST science. 
\item Additional pointers, such as weblinks, to additional more detailed and specialist information (e.g., software pipelines,
           Rubin image gallery, Rubin YouTube channel).
\item Contact list of media-trained cadre of scientists and engineers for interviews on specific topics

\end{itemize}

This Press Kit will be made available for download in a single collated, printable and linked pdf document,
as well as website text that is easy to navigate.


\subsection{Rubin Background Information} 

The Press Kit will also include extensive and appropriate textual background information.

The background information will be organized by general topic, and written in brief sentences that can be used as
sound-bites for in-person interviews. These are similar to key high-level messages discussed above, but are typically
longer and/or more detailed, and can be used as supplementary material to support the key messages.

Topics will include Rubin Observatory, Simonyi Survey Telescope, LSST Camera, Data Facility, LSST Software, LSST survey,
Science, People, Funding, Name, History of Rubin Observatory, Rubin Superlatives. This background information will
be made available in a single collated, printable and linked pdf document, as well as website text that is easy to navigate. 

  
\section{The Media Engagement Strategy}


\subsection{RFL Communication Channels}

The RFL impact will be maximized with maximized number of communication channels.
The channels under our control include
\begin{itemize}
\item Press releases
\item Expert interviews
\item Media visits 
\item AURA/Rubin/NOIRLab/NSF/SLAC/DOE social media incl. for live streaming of the main press conference
\item AURA/Rubin/NOIRLab/NSF/SLAC/DOE websites 
\item Auxiliary channels such as Reddit, Quora, and Ted Talks. 
\end{itemize} 

``External'' channels that not under our control include
\begin{itemize}
\item Print Media (for general public: newspapers, websites, etc.)
\item Broadcast media
\item Social media
\item Scientific trade publications (Sky \& Telescope, Scientific American etc.)
\end{itemize} 

All communication products will be created in both English and Spanish. 

Communication products are needed for, or build towards, the images and press release for the FL images.
They fall into some main categories listed below. Each of the deliverables will have their own more detailed planning documents. 


\subsection{Momentum building prior to RFL event}

Momentum building activities are important to build a following of people interested in the Rubin Observatory before the first light images are released. These activities also include educating the media and the public about the scientific goals of Rubin and its amazing technical achievements. 

Some momentum building activities include:
\begin{itemize} 
\item Organizational press releases focusing on construction milestones
\item Topical press releases focusing on the various science areas and featuring prominent Rubin scientists
\item Exhibits and interviews at winter AAS January 2025
\item Exhibits and interviews at summer AAS June 2025
\item General ramp-up of (reactive) ongoing in-person Media visits to the summit.
\end{itemize}


\subsection{Media engagement prior to RFL event}

Engaging with the media early and often will help ensure a successful RFL. The goal is to make it easy for the media
to gain access to ``behind the scenes'' information that they can use in their stories. Virtual tours and Q\&A panels
are a way to include those members of the media who cannot travel to the observatory in Chile. These events will
also build relationships between the communications team and the media for future coverage of the observatory. 

The process can be similar to NASA’s media engagement. A call to the media can be sent out via email and on social
media one week before the event. Those who want to attend the tour via zoom or telephone and ask questions will
need to register in advance. Those who do not register can still view the tour via, e.g., Rubin's YouTube channel. 

Some possible events include:
\begin{itemize} 
\item Virtual media tours
\item Social media days
\item Proactive in-person group media visits to the summit arranged in preselected slots with a curated experience
\item Virtual panels for Rubin Q\&A (Who: Rubin user/scientist, Rubin engineer, Rubin comms person; Why: opportunity
               to ask general questions)
\end{itemize} 



\subsection{Press conference}



Media events, especially embargoed ones, will help get the attention of the press for the RFL event.
Embargoes allow the press to prepare their stories ahead of the actual release date so when the images
are released the media stories are also ready to go. NSF has a press list of trusted media contacts that
can be the starting point for the invites to the embargoed press release. 


The key RFL media engagement steps will be executed in this order:  
\begin{enumerate}
\item Outreach to key media contacts
\item Embargoed “Get Ready for Rubin” Press conference: ``Background'' virtual press conference held in
conjunction with NSF, DOE, following: 
  \begin{itemize} 
    \item Embargoed release without images distributed one day prior
    \item Invite for this virtual press conference sent to trusted journalists
    \item Panel of experts will answer press questions
  \end{itemize}
\item Images released to embargo group 2 hours before general release
\item Primary Priority: Main in-person press conference day of image release
  \begin{itemize}
  \item  Hosted jointly by NSF and DOE in Washington DC (work with NSF to secure the National Press Club,
        which was the site of the Event Horizon Telescope and Gravitational Wave Optical Follow-up events)
  \item Invite to go out to embargo and non-embargo press and via social media
  \item Aim for noon Eastern to enable supporting events in Europe and further west, to Pacific time zone (see below) 
  \end{itemize}
\item Secondary Priority: Concurrent press conference in Santiago, in person and streamed from Primary site 
\item Tertiary Priority: press event could be streamed live to various other venues (see below)
\item Post-RFL follow-up media interviews
\end{enumerate} 

The ‘point-of-no-return’ RFL moment will be the date of the Media advisory announcement.
The decision about executing this step will undergo a review (SFL Media Alert Gate review).
%Criteria:
 


\subsection{Supporting hub events}

Given the exquisite preparations for the RFL and extensive resulting media products, it will be
relatively easily to organize an event concurrent with the press conference in D.C. The essential
driver is to use the main event to trigger and partially support a local event, which would also
showcase local Rubin aspects (construction and operations work, Science Collaborations, local
EPO intersests, etc.). The main press release text could be easily modified to allow room for
specific local content (including quotes) and promote local organizations. 

A number of groups have already expressed interest (a number of hubs in US, several hubs in
UK and France, also hubs from Italy, Hungary, Slovenia, Serbia, Croatia). The team should keep
them engaged and informed about progress, and send Press Kit and detailed timing for the
main press conference to them as soon as they are available. 



\subsection{Social networks}

In the lead up to the RFL image release it is important to build and engage with the social media audiences.
Many of our stakeholders are active in social media in addition to consuming traditional and new media.
A campaign that tells the story of Rubin and the science it will do will get our audience excited for the RFL images.

The team should have a brainstorming session to create the strategy and stories to tell. Some elements include:
\begin{itemize}
\item Strategy for lead up to RFL press release
\item Images
\item Text  (alt and main)
\item Graphics
\item Videos
\item Outreach to influencers
\end{itemize}



\subsection{Interviews and Spokespeople Preparation \label{interviews}} 

Pre-recorded short video interviews with selected Rubin team members and expert scientists will be included
in the Press Kit.

It is essential to provide media training for all people who will be identified for interviews. Even if they have
previous experience with the media, this training will inform them of Rubin specific messaging and topics to avoid.
After people for interviews have been selected, the team will provide to them
\begin{itemize}
\item Media training
\item Rubin slide deck (both for spokespeople and general members of the scientific community)
\item Main messages and talking points (consistent with key high-level messages, see Section~\ref{khlm})
\end{itemize} 



%\subsection{Web presence}


\subsection{Metrics for measuring success and their aspirational goals}

The following metrics have been used before, both by Rubin and other teams, to measure
the success of press releases and other media campaigns:
\begin{itemize}
\item Front page(s) of major newspaper(s) in US and Chile (goal: at least one in each country)
\item Near the top of trending on social media in the US and Chile (goal: at least in the top 5)
\item Meltwater’s ``number of theoretical readership'' metric for all articles covering the RFL
               should be in billions (over 4 billion to be comparable to the EHT metric).
\item Number of visits to main Rubin-related websites should increase by at least 50\%
\item Number of visits to Rubin's YouTube channel should increase by at least 50\%
\item A famous non-scientist talks about Rubin (e.g., President, ISS Astronaut, major influencer, etc.)
\item Several memes get significant attention on social media
\item A measureable impact on Google search trend
\item Mentioned in major TV news
\item Rubin is the google search image for day
\end{itemize}

Note that these goals, though based on prior experience, are aspirational since most are
beyond our direct control. The most meaningful metrics and measures of success will be
those that can be compared to other similar events (e.g. Event Horizon Telescope announcement
in 2019, see https://www.capjournal.org/issues/26/26\_11.pdf). 



  
\section{Team Organization, Roles \& Responsibilities, Schedule} \label{sec:team}

\begin{itemize}
\item - RCOC
\item - SFLcg
\item - 3 WGs
\item - decision making and approval process
\end{itemize}

Key stakeholders NSF, DOE, AURA, SLAC, NOIRLab, LSSTC, and AURA-O.

The following committees are relevant to the FL release (with leads and goals):

\begin{itemize}
\item Rubin Celebration Organizing Committee (RCOC):
\begin{itemize}
\item
An overarching committee designed to assure input from all key stakeholders, and progress reporting to all stakeholders, in the context of both FL release and two dedication ceremonies (the Simonyi Survey Telescope naming ceremony and the Rubin Observatory dedication ceremony)
\end{itemize}
\item System First Light Coordination Group (SFLcg), Zeljko Ivezic: the work is organized in three principal working groups, charged to design the FL content, produce it, package it for distribution, and distribute it to media and public:
\begin{itemize}
\item Images WG, Steve Ritz: planning the observing strategy and processing for the FL images
\item Communications WG, Ranpal Gill
\item EPO WG, Alan Strauss/Kristen Metzger
\item SFLcg is charged to coordinate the work of these three WGs, track their progress, and report it back to the RCOC.
\end{itemize}

These committees include staff from the Rubin Construction Project, Rubin Operations, representation from all stakeholders, and include experts from AURA HQ and NOIRLab’s CEE with relevant experience and expertise.

\end{itemize}

%\section{Links to detailed implementation documents} \label{sec:links}

\begin{itemize}
\item - detailed projectized deliverable list and schedule for each WG
\item - (embargoed) SFL data taking and data processing plans
\item - press kit contents (including Rubin background info and main science drivers)
\item - press conference organization
\item - media engagement details
\item - social networks strategy
\item - website and other support on the day of RFL event
\item - etc...
\end{itemize}


\subsection{Spokespeople Preparation}
Document: \href{https://docs.google.com/document/d/14pbmwzvYFVolCZJV7rvoWdZQHR82a8kKWZrBjAUKzzI/edit}{Rubin FL Media Training }

It is essential to provide media training for all people who will be identified for interviews. Even if they have previous experience with the media, this training will inform them of Rubin specific messaging and topics to avoid.

\begin{enumerate}
\item Media training for Rubin spokespeople
\item Rubin slide deck for spokespeople and general members of the scientific community
\item Main messages and talking points
\end{enumerate}

\subsection{Social Media Campaign}
Document: \href{https://docs.google.com/document/d/1-STd7aO3-_T73zZEHpxTrj2OtYoc_e6QERagPVNbfFQ/edit#heading=h.rkin5k30kjf}{Social Media Strategy for First Light}

In the lead up to the FL image release it is important to build and engage with the social media audience. Many of our stakeholders are active in social media in addition to consuming traditional and new media. A campaign that tells the story of Rubin and the science it will do will get our audience excited for the FL images. A brainstorming session to create the strategy and story will be planned. Some elements include:
\begin{enumerate}
\item Strategy for lead up to FL press release
\item Images
\item Text - alt and main
\item Graphics
\item Videos
\item Outreach to influencers
\end{enumerate}

\subsection{Media Engagement}
Engaging with the media early and often will help ensure a successful First Light images release. The goal is to make it easy for the media to gain access to “behind the scenes” information that they can use in their stories. Virtual tours and Q\&A panels are a way to include those members of the media who can’t travel to the observatory in Chile. These events will also build relationships between the communications team and the media for future coverage of the observatory.

The process can be similar to NASA’s media engagement. A call to the media can be sent out via email and on social media one week before the event. Those who want to attend the tour via zoom or telephone and ask questions will need to register in advance. Those who don’t register can still view the tour via YouTube or FaceBook.

Some possible events include:
\begin{enumerate}
\item Virtual media tours
\item Social media days
\item Proactive in-person group media visits to the summit arranged in preselected slots with a curated experience
\item Virtual panels for Rubin Q\&A
\begin{enumerate}
\item Who: Rubin user/scientist, Rubin engineer, Rubin comms person
\item Why: opportunity to ask general questions
\end{enumerate}
\end{enumerate}

\subsection{General Press Kit about Rubin Observatory}
Document: \href{https://docs.google.com/document/d/1aFM2bdE-ogtjh63nyE-UTsOtcQC9Ybis1s3LjQipL1Y/edit}{Rubin Media Kit Design Doc}

A Rubin Press Kit is essential and is required ASAP. Its purpose is to inform the media about Rubin Observatory and provide them with all the material they need to easily write stories and social media posts about Rubin, and ultimately about the FL images. The goal of the Press Kit is to make the media’s job covering Rubin as easy as possible.

Many publications no longer have dedicated science writers, so the information in the Rubin Press Kit must be ready for them to use “as is” and should be written for a non-science audience. That means the text should be free of jargon, and all science concepts must be explained. In addition, readily understandable comparisons or equivalencies are strongly recommended for the media who will be creating content for general audiences (e.g., one cell on the Sun is the size of Texas; one Rubin image is the size of 45 full moons).

The Press Kit will be a single document in PDF format.

\subsection{First Light Press Package Components}

The press package for the FL image release must be complete and easily accessible by media. The press package, along with the press kit, should provide the media covering Rubin First Light with everything they need for both traditional and social media products.

NOTE — all web servers must be prepared to accommodate high traffic.

First Light Press Package Contents
The contents of the first light press release package is listed below. Also a mock-up can be seen in \url{https://docs.google.com/presentation/d/1g1XEtyx2_pgcLC2sSWYDe6memaTYsjEaLmI6X6DzfiU/edit#slide=id.p}{this slide deck}.

\begin{enumerate}
\item Press release copy text
\item Images (in multiple formats)
\item Videos (in multiple formats, incl. fulldome)
\begin{enumerate}
\item Zooms
\item Pans
\item Dynamic videos (for instance RR Lyrae variables\footnote{Example: \url{https://www.eso.org/public/videos/eso1636a/}})
\item Video explainer, e.g.
\begin{enumerate}
\item \#SLAC-Explains
\item CosmoView
\item Reels
\end{enumerate}
\end{enumerate}
\item Image captions
\item Alt text
\item Video captions
\item Curated social-media-ready assets (graphics prepared for each platform, hashtags, etc)
\item Background photos and videos.
\begin{enumerate}
\item Three archives of high-end background images and videos have been set up
\item Also, it will be planned to have photographers on site to document the weeks of the FL observations.
\end{enumerate}
\item Contact list of media-trained cadre of scientists and engineers for interviews on specific topics
\item Links to
\begin{enumerate}
\item Background image and video galleries (well organized with multiple resolutions) with explainer videos, scientist interviews, SLAC camera videos etc.:
\begin{enumerate}
\item Rubinobs.org (\url{https://rubin.canto.com/v/gallery/album/HDSNU?display=curatedView&viewIndex=2})
\item \href{https://noirlab.edu/public/images/archive/category/rubin/}{NOIRLab.org} (\href{https://noirlab.edu/public/images/archive/category/rubin/}{image archive}, \href{https://noirlab.edu/public/videos/archive/category/rubin/}{video archive})
\item \href{http://NOIRLab.org}{Slac}
\end{enumerate}
\item “About Rubin” press kit PDF (see above)
\item Sonification product
\item Data in the Rubin science archive
\end{enumerate}
\end{enumerate}

\subsection{First Light Embargo policy}
Following the Information Sharing during Commissioning \cite{sitcomtn-076}.

To ensure the confidentiality and integrity of sensitive information within our collaboration prior to the official public FL release, the following embargo policy is in effect.

All AURA and SLAC staff as well as Rubin community members working on observations, data management, EPO, communications etc. shall adhere to the following rules of confidentiality.

All specific information about FL imaging products is deemed confidential until the embargo expires at the time of the FL press release. Specific information about FL images and targets can only be shared with the people involved in the FL campaign. Nothing specific about FL shall be shared outside, including on social media. General information about FL can be shared with the community, for instance at PCW. E.g. “The First Light observations will be taken over a 3-week period currently slated to start x March 2025. Several targets will be observed depending on their visibility at the time of observation, and their suitability to demonstrate various aspects of Rubin Observatory. Several committees consisting of experts are working on different aspects of First Light, in order to maximize the press and social media visibility.”

The Rubin Construction Director will maintain the list of personnel with access. A subset of staff working on the press release images will be the only ones who are granted access to image products.

Access to embargoed information is limited to essential personnel only. Secure methods for sharing information will be applied (e.g., encrypted emails, access controlled documents etc). All embargoed documents and images will be marked with "CONFIDENTIAL".

Embargoed information shall not be printed unless absolutely necessary. Any printed materials shall be stored in locked and secure locations. Printed materials should be disposed via secure shredding methods after the embargo period.

All media inquiries should be directed to the designated media liaisons. No staff or community members shall provide comments or information about FL to colleagues or to the media before the embargo lift date.

All suspected or actual breaches of this policy should be immediately reported to Rubin Construction Director. Violations of this policy may result in disciplinary actions, including termination of access to the FL project.

The Rubin Construction Director is responsible for enforcing this policy and ensuring compliance.

Exceptions to this policy can only be granted by the Rubin Construction Director in writing and after informing the SFLcg.





\end{document}




\appendix
% Include all the relevant bib files.
% https://lsst-texmf.lsst.io/lsstdoc.html#bibliographies
\section{References} \label{sec:bib}
\renewcommand{\refname}{} % Suppress default Bibliography section
\bibliography{local,lsst,lsst-dm,refs_ads,refs,books}

% Make sure lsst-texmf/bin/generateAcronyms.py is in your path
\section{Acronyms} \label{sec:acronyms}
\addtocounter{table}{-1}
\begin{longtable}{p{0.145\textwidth}p{0.8\textwidth}}\hline
\textbf{Acronym} & \textbf{Description}  \\\hline

DM & Data Management \\\hline
\end{longtable}

% If you want glossary uncomment below -- comment out the two lines above
%\printglossaries


